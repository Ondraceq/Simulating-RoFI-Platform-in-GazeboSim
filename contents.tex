%%%%%%%%%%%%%%%%%%%%%%%%%%%%%%%%%%%%%%%%%%%%%%%%%%%%%%%%%%%%%%%%%%%%
%% I, the copyright holder of this work, release this work into the
%% public domain. This applies worldwide. In some countries this may
%% not be legally possible; if so: I grant anyone the right to use
%% this work for any purpose, without any conditions, unless such
%% conditions are required by law.
%%%%%%%%%%%%%%%%%%%%%%%%%%%%%%%%%%%%%%%%%%%%%%%%%%%%%%%%%%%%%%%%%%%%

\documentclass[
  digital, %% The `digital` option enables the default options for the
           %% digital version of a document. Replace with `printed`
           %% to enable the default options for the printed version
           %% of a document.
%%  color,   %% Uncomment these lines (by removing the %% at the
%%           %% beginning) to use color in the digital version of your
%%           %% document
  table,   %% The `table` option causes the coloring of tables.
           %% Replace with `notable` to restore plain LaTeX tables.
  oneside, %% The `twoside` option enables double-sided typesetting.
           %% Use at least 120 g/m² paper to prevent show-through.
           %% Replace with `oneside` to use one-sided typesetting;
           %% use only if you don’t have access to a double-sided
           %% printer, or if one-sided typesetting is a formal
           %% requirement at your faculty.
  nolof,     %% The `lof` option prints the List of Figures. Replace
           %% with `nolof` to hide the List of Figures.
  nolot,     %% The `lot` option prints the List of Tables. Replace
           %% with `nolot` to hide the List of Tables.
  %% More options are listed in the user guide at
  %% <http://mirrors.ctan.org/macros/latex/contrib/fithesis/guide/mu/fi.pdf>.
]{fithesis3}
%% The following section sets up the locales used in the thesis.
\usepackage[resetfonts]{cmap} %% We need to load the T2A font encoding
\usepackage[T1,T2A]{fontenc}  %% to use the Cyrillic fonts with Russian texts.
\usepackage[
  main=english, %% By using `czech` or `slovak` as the main locale
                %% instead of `english`, you can typeset the thesis
                %% in either Czech or Slovak, respectively.
  english, german, russian, czech, slovak %% The additional keys allow
]{babel}        %% foreign texts to be typeset as follows:
%%
%%   \begin{otherlanguage}{german}  ... \end{otherlanguage}
%%   \begin{otherlanguage}{russian} ... \end{otherlanguage}
%%   \begin{otherlanguage}{czech}   ... \end{otherlanguage}
%%   \begin{otherlanguage}{slovak}  ... \end{otherlanguage}
%%
%% For non-Latin scripts, it may be necessary to load additional
%% fonts:
\usepackage{paratype}
\def\textrussian#1{{\usefont{T2A}{PTSerif-TLF}{m}{rm}#1}}
%%
%% The following section sets up the metadata of the thesis.
\thesissetup{
    date        = \the\year/\the\month/\the\day,
    university  = mu,
    faculty     = fi,
    type        = bc,
    author      = Ondřej Svoboda,
    gender      = m,
    advisor     = {prof. RNDr. Jiří Barnat, Ph.D.},
    title       = {Simulating RoFI Platform in GazeboSim},
    TeXtitle    = {Simulating RoFI Platform in~GazeboSim},
    keywords    = {robots, modular, simulation, GazeboSim, plugin, RoFIbot, RoFI, ...},
    TeXkeywords = {robots, modular, simulation, GazeboSim, plugin, RoFIbot, RoFI, \ldots},
    abstract    = {%
    },
    thanks      = {%
    },
    bib         = references.bib,
    assignment         = assignment.pdf,
}
\usepackage{makeidx}      %% The `makeidx` package contains
\makeindex                %% helper commands for index typesetting.
%% These additional packages are used within the document:
\usepackage{paralist} %% Compact list environments
\usepackage{amsmath}  %% Mathematics
\usepackage{amsthm}
\usepackage{amsfonts}
\usepackage{url}      %% Hyperlinks
\usepackage{markdown} %% Lightweight markup
\usepackage{tabularx} %% Tables
\usepackage{tabu}
\usepackage{booktabs}
\usepackage{listings} %% Source code highlighting
\lstset{
  basicstyle      = \ttfamily,
  identifierstyle = \color{black},
  keywordstyle    = \color{blue},
  keywordstyle    = {[2]\color{cyan}},
  keywordstyle    = {[3]\color{olive}},
  stringstyle     = \color{teal},
  commentstyle    = \itshape\color{magenta},
  breaklines      = true,
}
\usepackage{floatrow} %% Putting captions above tables
\floatsetup[table]{capposition=top}

\newcommand{\code}[1]{\texttt{#1}}

\begin{document}
\chapter{Introduction}
% What is the goal of this thesis

\section{Modular robots}

% What are modular robots (for those, who never heard of them)
% Why are they beneficial, what are advantages/disadvantages

what is a modular robot, benefits of modular robots

from modular robots, to self reconfigurable,
self-reconfigurable robots (what are they), why to use them;
- metamorph - number of same modules not capable of doing thing unless combined;
- word about smart matter

history of self-reconfigurable robots




\chapter{The RoFI platform}
% Description of the RoFI platform, what is it made of, capabilities
% what it can do, what it can't do




\section{RoFICoM}
\label{roficom}

\section{Universal module}
\label{univ-module}


\chapter{Robotic virtual simulator}
% What is a robotic simulator (for those, who never heard of it)
% Why is it beneficial, what are advantages/disadvantages

\section{Virtual simulator for modular robots}
% What is the difference against normal simulator

\section{Available modular robotic simulators}
% What robotic simulators exist
% Why do we need a new thing


\chapter{RoFI Simulator design}

\section{Platform selection}
% why Gazebo and not doing it from scratch, why not Unity or other simulator for modular robots

% TODO why exactly Gazebo

\section{Overall design}
% why client/server
% other posibility: compilating to plugin
% why one process/one rofi (e.g. SIGKILL)
% why HAL
% diagram about interfaces - where stands gazebo, plugins, models, user code, hal, ...

% pid controller
% messages

% TODO diagram of simulator design

\subsection{HAL interface}

\subsection{Gazebo integration}

\subsection{RoFICoM model}

\subsection{Creating modules}

\section{Challenges}
% difficulties in the design: what needs to be done, clients, connecting roficoms, saving world and reloading (saving plugin info to sdf), waiting

\subsection{Distributing the RoFI modules}

\subsection{Publishing messages from the client}

\subsection{Connecting RoFICoMs in simulation}

\subsection{Saving and loading worlds with state}

\subsection{Waiting in the client}


\chapter{Added utilities}

\section{World creator}
% short introduction about rofi descriptors, integration with them (world-creator)

\section{Lorris}
% and ping


\chapter{Experimental evaluation}
% examples, demos
% what works, what does not - how much does it not work

\section{Creating new modules}
\label{ex-modules}

% TODO

\section{Examples}

\subsection{Double wheel}

% TODO

\section{Creating worlds}

\subsection{Wheel}
% TODO

\subsection{Spider}
% TODO

\subsection{50 modules}
% TODO


\chapter{Conclusion}
% How I managed to make an awesome tool, that will help the world

\end{document}
